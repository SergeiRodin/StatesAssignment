\chapter*{Введение}\addcontentsline{toc}{chapter}{Введение}
\par Те кто занимается синтезом больших автоматов сталкивается со следующей проблемой. Техническое задание дается на языке абстрактных автоматов, тем или иным способом описываются переходы из состояния в состояние, выход в данном состоянии(например с помощью таблиц или диграмм перехода). Это описание поступает на техническую реализацию. К этой реализации переходят посредством кодирования состояний, входных символов и выходных символов и получаются описания, которые мы называем каноническими уравнениями. В этих уравнениях возникает блок переходов и блок выходов, у которых справа стоят булевские функции, т.е.кодирование приводит к возникновению булевских операторов. Затем булевские операторы реализуются в виде схем из функциональных элементов. Набор функциональных элементов может содержать(и на практике обычно содержит) функции многих переменных, операторы и т.д.(этот набор называется библиотекой). На основе этой библиотеки строят схемы. Сложность возникаемых схем зависит от всех этапов. 
%В данный момент опускается вопрос свзанный с оптимизацией схемы. 
В описанной процедуре реализации абстрактного автомата видно много степеней свободы. Однако первый и, на наш взгляд, очень важный этап это выбор кодирования состояний. После того как код выбран все остальные этапы детерминированы в известной степени. Можно, например, поработать над оптимизацией схемы в заданном базисе. Однако булевские операторы переходов и выходов уже заданы в некотором смысле.
%\par На практике часто необходимо решать задачу перехода от автоматного описания функционирования на язык схем. Например, при логическом синтезе чипов на первом этапе функционирование чипа описывается как конечный автомат.
%Переход к описанию на язык схем осуществляется с помощью кодирования алфавита состояний, входного алфавита и выходного алфавита в алфавите $\{0,1\}$.
%
%При этом важно выбрать кодирование, при котором достигается возможно меньшая сложность схемы.

\par С формальной точки зрения автомат --- это пятерка  $V=(A,Q,B,\varphi,\psi)$,
где  $A$ --- входной алфавит, $Q$ --- алфавит состояний, $B$ ---
выходной алфавит, $\varphi$ --- функция, которая по текущему входу и
состоянию определяет состояние автомата в следующий момент времени,
$\psi$ --- выходная функция, которая по текущему входу и состоянию
определяет выход автомата в текущий момент времени. Кодирование
алфавита состояний --- это отображение алфавита $Q$ в $E_2^k$, при
котором каждому состоянию из $Q$ ставится в соответствие вектор из
$E_2^k$. Кодирование входного алфавита --- это отображение алфавита
$A$ в~$E_2^p$, при котором каждому элементу из $A$ ставится в
соответствие вектор из $E_2^p$. Кодирование выходного алфавита ---
это отображение алфавита $B$ в~$E_2^l$, при котором каждому элементу
из $B$ ставится в соответствие вектор из $E_2^l$. Кодирования
алфавита состояния, входного алфавита и выходного алфавита порождают
булев оператор $\phi: E_2^{k+p}\rightarrow E_2^{k+l}$, где $p$ ---
длина кодового набора для символов множества $A$, $k$ --- длина
кодового набора для символов множества $Q$, $l$ --- длина кодового
набора для символов множества $B$.

Оператор $\phi$ можно рассматривать как набор $k+l$ булевых функций
от $n+k$ переменных. При этом важно выбрать кодирование, при котором достигается возможно меньшая сложность схемы. 



Вопрос <<простой>> реализации автоматов был рассмотрен в работах 60-х годов Стирнса и Хартманиса. В основе исследований лежал метод пар разбиений, который был в дальнейшем трансформирован в метод алгебры пар. Покрытием $\pi$ множества $Q$ называется разбиение множества $Q$ на непересекающиеся подмножества, которые называются блоками \cite{hartmanis_stearns_feedback_en}. На множестве покрытий вводится частичный порядок. Говорим, что покрытие $\pi_1\ge\pi_2$, если каждый блок покрытия $\pi_2$ полностью содержится в некотором блоке $\pi_2$. С помощью порядка вводится операции сложения и умножения на покрытиях. По определению $\pi_1+\pi_2$ наименьшая верхняя грань покрытий $\pi_1$ и $\pi_2$, а $\pi_1\cdot\pi_2$ наибольшая нижняя грань покрытий $\pi_1$ и $\pi_2$. Упорядоченная пара $(\pi,\tau)$ покрытий множества $Q$ автомата ${\cal A}$ образует пару покрытий, если для любых двух состояний $q_i$ и $q_j$, лежащих в одном блоке покрытия $\pi$, и для любого входа $a\in A$, состояния в следующий момент времени $\varphi(a,q_i)$ и $\varphi(a,q_j)$ лежат в одном блоке покрытия $\tau$. На основании понятий пары покрытий и введенного порядка для заданного покрытия $\pi$ определяется <<наименьшее>> покрытие $\tau$, составляющее пару покрытий $(\pi, \tau)$, такое покрытие обозначается $m(\pi)$. Для покрытия $\pi$ наибольшее покрытие $\tau$, такое что $(\tau,\pi)$ есть пара покрытий обозначается $M(\pi)$
С каждым кодированием алфавита состояний $Q$ векторами длины $k$ можно связать $k$ покрытий множества $Q$, пусть $i$-ой переменной соответствует покрытие $\pi_i$. Было доказано, что переменная, соответствующая покрытию $\tau$, зависит от некоторого подмножества переменных с индексами $P\subset {0,\ldots,k-1}$ тогда и только тогда, когда $(\prod_{i\in P}\pi_i,\tau)$ есть пара покрытий. Существование кодирования с таким свойством эквивалентно существованию разложению(разбиению,декомпозиции) автомата на меньшие автоматы.
Введенные понятия также могут быть использованы для изучения обратной связи в реализации автомата. С помощью разбиения множества вводится формализация переменной, участвующей в операции обратной связи. Разбиение $\pi_f$ множества состояний называется разбиением обратной связи, если существует набор покрытий $\{\tau_i\}(1\le i\le k)$ таких,что $\prod_{i=1}^k\tau_i$ и $(\pi_f\cdot\prod_{i<j},\tau_j)$ есть пара покрытий, $j=1,\ldots,k$. Для покрытия $\pi$ определен ряд покрытий $A^i(\pi)$ по следующему правилу $A^1(\pi)=\pi$ и $A^{i+1}(\pi)=\pi\cdot m(A^i(\pi))$. Если используется кодирование кодами длины $n$ обозначим $A(\pi)=A^n(\pi)$. С помощью введенных понятий сформулирован критерий, для автомата существует реализация, где переменная соответствующая покрытию $\pi_f$, используется в операции обратной связи тогда и только тогда, когда $A(\pi_f)=0$.
Как следствие данного результата доказан критерий используется ли операция обратной связи в реализации автомата, а именно операция обратной свzзи не используется тогда и только тогда, когда $m^n(I)=0$ или, что эквивалентно $M^n(0)=I$

Также данная техника в отдельных случаях позволяет находить разложения автомата в последовательно-параллельное соединение двух автоматов, в одном из которых не используется операция обратной связи.

Как видно предложенная техника является довольно мощной, с одной стороны, она позволяет абстрагироваться от конкретной реализации автомата, с другой стороны позволяет находить декомпозицию автомата на более простые автоматы. В ее основе лежит понятие пары покрытий и введенные операции сложения и умножения. Данный подход может быть обобщен до рассмотрения пар объектов(не обязательно покрытий), удовлетворяющих ряду свойств. Таким образом возникло понятие алгебры пар. А именно, пусть $L_1$ и $L_2$ - конечные структуры, тогда подмножество $\Delta$ прямого произведения $L_1\prime L_2$ является алгеброй пар тогда и только тогда, когда выполнены свойства:
\begin{enumerate}
\item Если $(x_1,y_1),(x_2,y_2)\in\Delta$, то $(x_1\cdot x_2,y_1\cdot y_2)\in \Delta$ и $(x_1+x_2,y_1+y_2)\in \Delta$
\item $\forall x\in L_1$ и $\forall y\in L_2$, $(x,I)\Delta$ и $(0,y)\in\Delta$.
\end{enumerate}
Под операцией умножения понимается операция взятия наибольшей нижней грани, под операцией суммы понимается наименьшей верхней грани.
Был доказан ряд результатов, связывающий возможность декомпозиции автомата в последовательно-параллельное соединение меньших автоматов с наличием пар покрытий множества состояний с определенными свойствами \cite{stearns_hartmanis_main_ru}.

Хотя данный подход и позволяет находить более простую реализацию автомата, он не позволяет определить сложность возникаемого оператора. Сложность такого оператора можно определить как максимальную сложность получающихся булевых функций. Как известно \cite{Yablonski}, каждой булевой функции единственным образом
соответствует полином Жегалкина. Мы будем понимать сложность оператора как максимальную из сложностей полиномов Жегалкина функций, задающих этот оператор, т.\,е. как максимальную степень полиномов, а сложность автомата --- как сложность оператора $\phi$. Таким образом,установив связь между автоматом, кодировкой и возникающими полиномами, можно найти минимальную сложность реализации автомата.

\par Естественно начинать такого рода исследования с <<простейших>>, линейных полиномов. Одной из основных задач работы было изучение автоматов, у которых существует кодирование, такое что получаемые при данном кодировании, булевские функции являются линейными. Т.е. вопрос является ли автомат изоморфным линейному автомату. 

Наиболее полно линейные автоматы изучены в книге А. Гилла <<Линейные последовательностные машины>> \cite{gill}. Линейным автоматом называется ${\cal L}=(E_p^r, E_p^k, E_p^m,\varphi,\psi,q_0)$, где $\varphi(x,q)=Ax+Bq, \psi(x,q)=Cx+Dq$, где сумма понимается в смысле суммы в поле Галуа $GF(p)$, где p --- простое число.
В книге введены и изучены понятия эквивалентности, подобия, минимальности, канонической формы, управляемости и предсказуемости линейных автоматов. Отдельно рассмотрены и изучены автономные линейные автоматы и автоматы с нулевым начальным состоянием. Для каждого из классов изучены проблемы анализа и синтеза автоматов. Также отдельно рассматриваются процессы изменения состояний автоматов и развивается теория множества циклов и деревьев.

И в частности рассмотрен вопрос линейной реализуемости для автоматов, у которых множество входных сигналов --- это множество $l$-мерных векторов над полем $GF(p)$,  множество выходных сигналов --- это множество $m$-мерных векторов над полем $GF(p)$, множество состояний содержит $p^n$ элементов. Для таких автоматов доказаны следующие утверждения. Пусть задано множества $S=\{s_1,s_2,\ldots,s_{n-1}\}$ и $y_0^j(t)$ --- выход автомата в момент $t$ при входном символе $0$ и начальном состоянии $s_j$. Введем матрицу
$$
L=\begin{pmatrix}
y_0^1(0) && y_0^2(0) && \ldots &&  y_0^{p^n}(0) \\
y_0^1(1) && y_0^2(1) && \ldots &&  y_0^{p^n}(1) \\
\ldots \\
y_0^1(n-1) && y_0^2(n-1) && \ldots &&  y_0^{p^n}(n-1) \\
\end{pmatrix}
$$
Составим из первых $n$ линейно независимых строк матрицы $L$ матрицу $\widetilde{L}$, а через $s_j$ обозначим $j$-й вектор-столбец матрицы $\widetilde{L}$. Тогда, если существует изоморфный исходному линейный автомат ${\mathfrak A}$, состояние $s_j$ заданного автомата эквивалентно состоянию ${\bf s}_j$ автомата ${\mathfrak A}$
У исходного автомата можно переобозначить каждое состояния $s_j$ на ${\bf s_j}$ и пусть $\delta$ и $\lambda$ его функция переходов и выходов.
Введем обозначения 
$$
{\bf s'}_i=\delta({\bf s}^i,{\bf 0}),\quad {\bf s''}_i=\delta({\bf 0},{\bf u}^i),
$$
$$
{\bf y'}_i=\lambda({\bf s}^i,{\bf 0}),\quad {\bf y''}_i=\lambda({\bf 0},{\bf u}^i).
$$
Верно утверждение, если существует изоморфный исходному линейный автомат ${\mathfrak A}$,
$$
A=||{\bf s'}_1,{\bf s'}_2,\ldots,{\bf s'}_n||,
$$
$$
B=||{\bf s''}_1,{\bf s''}_2,\ldots,{\bf s''}_l||,
$$
$$
C=||{\bf y'}_1,{\bf y'}_2,\ldots,{\bf y'}_n||,
$$
$$
D=||{\bf y''}_1,{\bf y''}_2,\ldots,{\bf y''}_l||.
$$
Таким образом проверка линейной релизуемости автомата осуществляется проверкой, что состояние автомата в следующей момент времени вычисляется как $A{\bf s}+B{\bf u}$, а выход автомата определяется выражением $C{\bf s}+D{\bf u}$ для всех ${\bf s}$ и ${\bf u}$. Если это не верно, то исходный автомат не является линейно реализуемым. Однако, было замечено, что исходный автомат может превратиться в линейно реализуемый, если переобозначить входные и(или) выходные символы.
Таким образом, данный результат не в полной мере решает вопрос о линейной реализуемости автомата.

Для автомата можно ввести понятие внутренней полугруппы.
Внутренняя полугруппа определяется как замыкание отображений
множества состояний в себя, определяемых входными символами
\cite{arbib}. Таким образом, на переходную систему автомата можно
смотреть как на набор отображений, и линейная реализуемость автомата
определяется линейной реализуемостью этих отображений.

В работе Экера \cite{ecker} рассмотрен вопрос о линейно реализуемых автоматах с точки зрения внутренней полугруппы.   
В частности приведена характеризация внутренней полугруппы линейно реализуемого автомата в случае, когда внутренняя полугруппа автомата --- группа, а именно, группа $G$ изоморфна внутренней полугруппе линейного над полем $GF(p)$ автомата тогда и только тогда, когда:
\begin{enumerate}
\item $G$ содержит нормальную, абелеву подгруппу $N$,у которой все элементы кроме единичного имеют порядок $p$;
\item существует такой элемент $c\in G$, что $N$ и $c$ образуют $G$.
\end{enumerate}
Однако, данная теорема не в полной мере решает вопрос о линейно реализуемости автомата, так автомат, внутренняя полугруппа которого удовлетворяет условиям теоремы, не всегда является линейно реализуемым, что было показано в работе Хартманиса и Уолтера \cite{hartmanis_walter}. В этой же работе развит подход, предложенный Экером, и сформулирован критерий линейной реализуемости автомата в случае, когда автомата --- перестановочный и сильно-связный. Также отметим, что в работе рассмотриваются автоматы без выходов или переходные системы. В случае, когда внутренняя полугруппа автомата $V$ --- это транзитивная группа $G$, состояния автомата $V$ могут быть рассмотрены как левые смежные классы группы $G$ по простой подгруппе $H$, т.е. переходная система автомата с точностью до изоморфизма определяется группой $G$, простой подгруппой $H$ и подмножеством $I\in G$, где $G$ --- внутренняя полугруппа, множество смежных классов --- множество состояний, подмножество $I$ --- множество входов. Переходная система, задаваемая этими объектами, обозначается $V_{G,H,I}$. Тогда результат может быть сформулирован следующим образом, переходная система $V_{G,H,I}$ - линейно реализуема над полем $GF(p)$ тогда и только тогда, когда
\begin{enumerate}
\item $G$ содержит нормальную, абелеву подгруппу $N$, у которой все элементы кроме единичного имеют порядок $p$;
\item существует такой элемент $c\in G$, что $N$ и $c$ образуют $G$;
\item $N\cap H={e}$;
\item $I\subseteq Na$ для некоторого $a\in G$
\end{enumerate}
Данный результат ограничен перестановочными сильно-связными автоматами. Дальнейшее развитие изложенный подход получил в работе Хартфила и Максона \cite{hartfiel_maxson}. В работе сформулированы две теоремы. Первая --- это обобщение результата Экера на полугрупповой случай, а именно существует изоморфизм $\beta$ между полугруппой $S=<\phi_0,\phi_1,\ldots,\phi_r>$ и внутренней полугруппой автомата, линейно реализуемого над полем $GF(p)$ тогда и только тогда, когда $S$ --- подполугруппа моноида $J=<\phi_0,N>$, где
\begin{enumerate}
\item $N$ --- абелева группу, содержащая единичный эелемент $J$, у которой все элементы имеют порядок $p$.
\item $\phi_0 N=N\phi_0$,
\item если для $\phi',\phi''\in N$ верно $\phi'\phi_0^k=\phi''\phi_0^k$, то $\phi'=\phi''$
\item $\{\phi_0,\phi_1,\ldots,\phi_r\}\subseteq N\phi_0$
\end{enumerate}
Вторая теорема обобщает результат Хартманиса и Уолтера на случай полугрупповых автоматов. Пусть задана полугруппа(моноид) $S$, тогда автомат имеющий в качестве внутренней полугруппы полугруппу $S$ определяется, с точность до изоморфизма, полугруппой $S$, множеством $I\in S$, и левой конгруэнтностью $\rho$ на $S$, где $I$ --- множество входов, классы эквивалентности, определяемые конгруэнтностью $\rho$ --- множество состояний, умножение в полугруппе $S$ определяет функцию переходов. Такой автомат обозначен через $V_{S,I,\rho}$. 
В определенных терминах сформулирован критерий линейно реализуемости, а именно автомат $V_{S,I,\rho}$ линейно реализуем над полем $GF(p)$ тогда и только тогда, когда $S$ --- подполугруппа моноида $J=<\phi_0,N>$, где
\begin{enumerate}
\item $N$ --- абелева группу, содержащая единичный элемент $J$, у которой все элементы имеют порядок $p$.
\item $\phi_0 N=N\phi_0$,
\item если для $\phi',\phi''\in N$ верно $\phi'\phi_0^k=\phi''\phi_0^k$, то $\phi'=\phi''$
\item $\{\phi_0,\phi_1,\ldots,\phi_r\}\subseteq N\phi_0$
\item конгруэнтность $\rho$ может быть доопределена до левой конгруэнтности $\bar{\rho}$ на $J$ таким образом, что $\rho\cap\mu=id$, где $\mu$ --- конгруэнтность на $J$, определяемая правилом: $s\mu t$ если существует такой элемент $\psi\in N$, что $s=\psi t$.
\end{enumerate}
%Summary
%First the problem is solved how one can decide whether an arbitrary finite semigroup H is linearly A-realizable, i.e., whether there exists a linearly realizable finite automaton having a semigroup isomorphic to H. This leads to a question about the existence of certain generating subsets of H. The determination of these subsets is rather complicated in case H-HH=Ø and very simple in case H-HH#Ø. But in the first case we are able to clear up completely the structure of the semigroups which are linearly A-realizable: These are exactly the finite right groups which have maximal subgroups of the type described by Ecker in [4]. In the second case we get only necessary structure conditions. Among other things we shall see: If a semigroup H is linearly A-realizable one can define a congruence relation ρ on it having the property, that H is isomorphic to a semigroup of a strongly connected and linearly realizable automaton iff the so-called index of H equals the index of H/ρ. Developing these results about semigroups we obtain at the same time many structure theorems about linearly realizable automata.

\par Вообще говоря, возникаемые при кодировании операторы, являются частично определенными, так как значение операторов определено только на кодирующих наборах. <<Правильное>> доопределение может как <<упростить>> операторы, так и <<усложнить>> их.  В первой главе работы будут изучаться автоматы мощность множества состояний есть степень $2$ и неизбыточные кодирования. Зная какие переходные системы имеют линейную реализацию, можно доопределять частично определенные операторы до линейных или установить, что это невозможно.
\par Во второй главе работы будут рассмотрены автоматы с произвольным числом состояний и избыточные кодирования.
\par В первом разделе каждой главы будут рассмотрены вопросы линейной реализуемости подстановок. Во втором разделе будет изучатся линейная реализуемость переходной системы автомата. В третьем разделе будут приведены результаты для автомата.
%В данной работе будут изучаться автоматы мощность множества состояний есть степень $2$ и неизбыточные кодирования. Зная какие переходные системы имеют линейную реализацию, можно доопределять частично определенные операторы до линейных или установить, что это невозможно.
%\par В первом разделе будут рассмотрены вопросы линейной реализуемости подстановок. Во втором разделе будет изучаться линейная реализуемость переходной системы автомата. В третьем разделе будут приведены результаты для автомата.

В {\bf заключении} представлены основные результаты диссертации.

\subsubsection*{Благодарности}
Автор выражает глубокую благодарность своему научному руководителю ---
профессору Станиславу Владимировичу Алёшину за постановку задачи,
обсуждение результатов и постоянное внимание к работе.
Автор благодарен всем сотрудникам кафедры Математической теории
интеллектуальных систем Механико-математического факультета МГУ,
в особенности заведующему кафедрой профессору Валерию Борисовичу \mbox{Кудрявцеву},
за поддержку работы и творческую атмосферу на кафедре.

